\chapter{Concluding Remarks}

\section{Summary and Contributions}

This thesis described three major problems in population genetics and genomics and discussed proposed solutions to solve them. For each approach a solution utilizing machine learning was proposed that performs better than previous state of the art for all tasks compared to.

The key takeaway from these works is that population genetics has a lot of room to grow, both in terms of data acquisition and methodologies. Sequencing continues to become more affordable, new paradigms of sampling and analysis are implemented as time goes on, and yet the field continues to improve its understanding without hitting any limits. Deep learning approaches in population genetics are not the end-all be-all, however they do indicate a larger shift towards data-driven approaches to the field that can fundamentally alter how researchers develop models in increasingly robust ways.

\section{Challenges and Limitations}

Despite the success of data-driven approaches, much of the conclusions that can be made about population genetics are based on assumptions. This is an issue inherent to the field, one cannot go back in time to truly capture historical data, and a complete reconstruction of all information would be infeasible even if time travel was indeed possible. These assumptions are mitigated by good experimental design and analysis, but will always be couched in the understanding that simulation-based and inference-based methods work on assumptions that can always be broken. Any computational challenges currently faced today can be overcome with time and engineering, but the perspective from which we view genetics cannot.

\section{Future Directions}

Population genetics will become more tightly integrated with other biological domains as its relevance and accuracy of its techniques increases. Understanding the \textit{how} using population genetics does not lead to understanding the \textit{why} learned from fields such as molecular biology or biochemistry. Combining understanding from population genetics with other fields will increase the understanding of how evolution works at all scales, and will complement all associated fields with increased context.

Along with this, foundational models will be \textit{in vogue} for the forseeable future. Multiple companies and research groups are attempting to create "foundation models for biology", which are AI models that are trained on massive amounts of data to gain an implicit understanding of a domain, and are then fine-tuned for specific tasks. They are more flexible, and can be more accurate than the more traditional machine learning approach, however some domains struggle to see the benefit. Population genetics will see at least a few of these foundation models pop up, and they are likely to be useful in a variety of contexts within the field, however their utility across a domain as wide as population genetics is yet to be seen. 